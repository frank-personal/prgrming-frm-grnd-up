\documentclass{article}

% packages
% 1. basic formatting/usage packages
\usepackage [utf8]{inputenc}
\usepackage {amsmath, amssymb, amsthm}
\usepackage {color}
\usepackage {enumerate}
\usepackage {bm}
\usepackage {verbatim}

% 2. packages for including images
\usepackage {graphicx}
\graphicspath { {./img} }

% 3. package for removing indents
\usepackage [indent=0em]{parskip}

% 4. package for reducing margins
\usepackage [margin=0.75in]{geometry}

% function definitions
% 1. monospace font
\newcommand {\code}{\texttt}

\title {Chapter 2: Computer Architecture}
\author {Frank Vojtesek}

\begin {document}

	\maketitle
	
	Modern computer archictecture is based off of the \textbf {Von Neumann Architecture},
	which:
	\begin {itemize}
		\item Divides the computer up into two parts: (1) the Central Processing Unit (CPU),
			and (2) memory
		\item Specifies that not only computer data should live in memory, but also the
			programs which control the computers operation 
	\end {itemize}

	\section {Structure of Computer Memory}
	\label {sec:structure_of_memory}	
	
	\textbf {Computer Memory} is one or many physical chips on the computers mother board.
	
	Internally it is organized as a numbered sequence of fixed size containers, each
	container contains a number.
	\begin {itemize}
		\item E.g., if a computer has 256 megabytes of memory, then your computer
			contains roughly 256 million fixed size storage locations
		\item Each storage location contains 1 byte (8 bits) of information
		\item I.e., a positive integer whose value is between \code 0 and \code {255}
			($ 2^8 - 1$), usually
			represented in binary, e.g., \code {10011101}
	\end {itemize}
	
	All results of any computation performed on the computer is stored in memory.
	
	All instructions for the CPU to follow are stored in memory also.
	
	The only difference between the data and the functions is the context (information)
	that the CPU uses to understand the numbers stored in memory.

	\section {The CPU}
	\label {sec:the_cpu}	
	
	The CPU takes care of interpretting the data and instructions stored in memory.
	
	The CPU reads in instructions from memory one at a time and excutes them,
	this is called the \textbf {fetch-execute cycle}.
	
	The CPU consists of the following elements:
	\begin {itemize}
		\item Program Counter
		\item Instruction Decoder
		\item Data bus
		\item General-purpose registers
		\item Arithemetic and logic unit
	\end {itemize}
	
	The \textbf {program counter} is used to tell the computer where to fetch the next
	instruction.
	
	The program counter holds the memory address of the next instruction to be executed.
	
	The CPU begins by looking at the program counter, and fetching whatever number is
	stored in memory at the location specified.
	
	This number is then passed on to the \textbf {instruction decoder} and figures out
	what process needs to take place, and what memory locations are going to be involved in
	this process.
	\begin {itemize}
		\item These actions could include \textbf {addition}, \textbf {subtraction},
			\textbf {multiplication}, \textbf {data movement}, etc...
		\item Computer instructions usually consist of both the actual instruction
			and the list of memory locations that are used to carry it out
	\end {itemize}
	
	Now the computer uses the \textbf {data bus} to fetch the memory locations used
	in the calculation
	\begin {itemize}
		\item The data bus is the connection between the CPU and memory
		\item It is thye actual wire that connects them
		\item If you look on the motherboard of a computer, the wires that
			go in and out from the memory are the data bus
	\end {itemize}
	
	In addition to the memory outside of the processsor, the processor itself has
	some special, high-speed memory locations called registers
	\begin {itemize}
		\item there are two kinds of registers: (1) \textbf {general registers}, and
			(2) \textbf {special-purpose registers}
		\item General registers are where most of the processing happens: e.g., addition,
			subtraction, multiplication, comparisons, etc...
		\item Computers have very few general purpose registers, most information
			is stored in the main memory, and brought into registers for processing, and
			then put back into memory when the processing is complete
		\item Special-purpose registers are registers which have very specific 
			purposes, these will be discussed more when they come up
	\end {itemize}
	
	Once the CPU has retireved all of the data it needs, it passes on the data
	and decoded instruction to the \textbf {arithmetic and logic unit}, where the
	instruction is actually executed.
	
	After the results of the computation have been calculated, the results are placed
	on the data bus and sent to the appropriate location in memory or a register, as 
	specified by the instruction.
	
	This description is simplified, and actual processors are quite a bit more complicated,
	more information can be found in the source text.
	
	\section {Some Terminology}
	\label {sec:terms}
	
	Computer memory is a numbered sequence of fixed-size storage locations.
	
	The number attached to each storage location is called it's
	\textbf {memory address}.
	
	The size of a single storage location is called a \textbf {byte}.
	
	On \code {x86} processors, a byte is a number between \code {0} and \code {255}.
	
	You may be wondering how computers can display and use text, graphics, and even
	large numbers when all they can do is store numbers betwen \code 0 and \code {255}
	
	The computer has specialized hardware like a graphics card with special interpretations
	of each number
	\begin {itemize}
		\item When displaying to the screen, the computer uses ASCII code tables to
			translate the numbers you are sending into letters to display on the screen
		\item In the case of ASCII, one number translates to exactly on character (unicode
			makes things a little bit more complicated)
		\item For example, the capital letter \code {'A'} is represented by the number
			\code {65} (\code {01000001}), the number character \code {'1'} is 
			represented by the number \code {49} (\code {00110001})
		\item In order to print out \code {"HELLO"}, you would need to give the computer
			the sequence of numbers \code {72, 69, 76, 76, 79}
		\item To print out the number \code {"100"}, you would need to give the computer
			the sequence of numbners \code {49, 48, 48}
	\end {itemize}
	
	In order to represent numbers larger than \code {255} we use combinations of bytes
	\begin {itemize}
		\item Suppose we wanted to represent the number \code {55642}, we could do this
			using two bytes (16 bits)
		\item The number \code {55642} is 
			{\color {red}\code {11011001}}{\color {blue}\code{01011010}} 
			in binary, a 16-bit
			number
		\item We can represent it with the two 8-bit numbers by storing the top 8-bits
			in one memory address, and the bottom 8-bits in another
		\item So, we would be able to represent \code {55642} with \code {217}
			({\color {red}\code {11011001}}) and \code {90} 
			({\color {blue}\code{01011010}})
		\item These two bytes written next two eachother make up a single 16-bit integer
		\item The CPU has hardware to handle the mathematics that goes into performing
			operations on combinations of bytes
	\end {itemize}
	
	We mentioned earlier that in addition to regular memory, the computer has special
	purpose high-speed memory locations called \textbf {registers}.
	
	Registers keep the contents of the numbers that you are currently manipulating.
	
	On the computers we are using, registers are each four bytes long.
	
	The size of a typical register is called the computer's \textbf {word size},
	\code {x86} processors have four-byte words.
	
	This means that it is most natural for computers to do computations 4 bytes at a time.
	
	Memory addresses are also four bytes (1 word) long, and therefore also fit into
	a register.
	
	Notice that this means that we can store addresses the same way we store any other
	number.
	
	In fact, the computer can't tell the difference between a value that is an address,
	a value that is a number, a value that is an ASCII code, or a value that you have
	decided to use for another purpose.
	
	A number becomes an ASCII code when you attempt to display it, and a number becomes
	a memory address when you attempt to lookup the byte it points to.
	\begin {itemize}
		\item This is crucial to understanding how a computer works
		\item The idea is based around the difference between data and information
		\item The number itself stored in memory has no meaning without the context
			(information, instructions) on how to process it
		\item For example, consider the number: 
		\begin {center}
			\code {0100100001000101010011000100110001001111}
		\end {center}
		\item On it's own it is meaningless, however if I told you that this 
			number represents a string in ASCII, and that every 
			8 digits (8-bits) represents an ASCII code (a number between 
			\code 0 and \code {255}), then we could parse the number as follows
		\item First we break the number into 8 digit (bit) chunks:
			\begin {center}
				\code {01001000 01000101 01001100 01001100 01001111}
			\end {center}
		\item Then we can work out the ASCII character that corresponds to each
			number as follows
			\label {tab:ascii_conversion_ex}
			\begin {center}
				\begin {tabular}{ccc}
					\hline
					\textbf {Binary Number}
						& \textbf {Decimal Number}	
						& \textbf {ASCII Character}	
					\\
					\hline
					\code {01001000}	&	\code {72}		&	\code {'H'}		\\
					\code {01000101}	&	\code {69}		&	\code {'E'}		\\
					\code {01001100}	&	\code {76}		&	\code {'L'}		\\
					\code {01001100}	&	\code {76}		&	\code {'L'}		\\
					\code {01001111}	&	\code {79}		&	\code {'O'}		\\
					\hline
				\end {tabular}
			\end {center}
		\item And we see that the number represents the word \code {"HELLO"} in 
			ASCII that was mentioned earlier
	\end {itemize}
	
	 Addresses which are stored in memory are also called \textbf {pointers}, because
	 instead of representing a regular value, they point to a different location in 
	 memory.
	 
	 As we've mentioned earlier, computer instructions are alos stored in memory.
	 
	 In fact, they are stored exactly the same way that other data is stored.
	 
	 The only way the computer knowsd that a memory location is an instruction is that a
	 special-purpose register called the \textbf {instruction pointer} points to them
	 at one point or another.
	 
	 If the instruction poihnter points to a memory word, it is loaded as an instruction.
	 
	 Other than that, the computer has no way of knowing the difference between programs
	 and other types of data.
	 \begin {itemize}
	 	\item The source text points out here that there are some types of processers
	 		and operating systems that mark the regions of memory that can be executed
	 		with a special marker that indicates this.
	 \end {itemize}
	 
	 \section {Interpreting Memory}
	 \label {sec:interpreting_memory}
	 
	 A computer has no idea what your program is supposed to do, it will do exactly what
	 you tell it to do.
	 
	 If you accidentally print out a regular number instead of the ASCII codes that make
	 up the number's digits, the computer will try to look up what the number represents
	 in ASCII and print that.
	 
	 If you tell the computer to start executing instructions at a location containing
	 data instead of program instructions, who knows how the computer will interpret
	 that, but it will certainly try.
	 
	 The computer will execute your instructions in the exact order that you specify,
	 even if it doesn't make sense.
	 \begin {itemize}
	 	\item Consider the previous example with the binary number representing the
	 		string \code {"HELLO"}
	 	\item If I told you that you had to first invert every bit in the number,
	 		and then every 8 digits would represent and ASCII Character
	 	\item Then we would first invert the bits of the number, and then split the digits
	 		into 8-bit groups:
	 		\begin {center}
				\code {10110111 10111010 10110011 10110011 10110000}
			\end {center}
		\item And decode the message as \code {"·º³³°"} which doesn't make any sense 
	\end {itemize}
	
	The computer will do exactly what you tell it to, no matter how little sense it
	makes, so you need to know exactly how to have your data in memory.
	
	Computers can only store numbers, so letters, pictures, music, web pages, documents,
	and everything else are just long sequences of numbers in the computer, which
	particular programs know how to interpret.
	\begin  {itemize}
		\item For example, say that you wanted to store customer information in memory
		\item One way to do so would be to set the maximum size for the customer's 
			name and address to 50 ASCII characters each, which would be 
			50 bytes for each
		\item Supposethat  you represented the customer age and the customer id 
			with an integer (4 bytes each).
		\item If the pointer to the start of the record was \code {0x6dfed4} in 
			hexadecimal, then the record could be described as follows:
			\begin {center}
				\label {ref:fixed_width_data}
				\begin {tabular}{ll}
					\hline
					\textbf {Field Name} 	&	\textbf {Record Start Pointer}	\\
					\hline
					\code {customerName}		&	\code {0x6dfed4}					\\
					\code {customerAddress}	&	\code {0x6dfed4 + 50 bytes}		\\
					\code {customerAge}		&	\code {0x6dfed4 + 100 bytes}		\\
					\code {customerId}		&	\code {0x6dfed4 + 104 bytes}		\\
					\hline	
				\end {tabular}
			\end {center}
		\item This will cause issues if the \code {customerAddress} for example, is
			more than 50 characters, to illustrate this issue, let's consider a simpler
			example
		\item Consider a simple record of the form:
			\begin {center}
				\begin {tabular}{lll}
					\hline					
					\textbf {Field Name}	& \textbf {Data Type}	& \textbf {Length}	\\
					\hline
					\code {initials}		& \code {ASCII String}	& \code {2 bytes}	\\
					\code {age}			& \code {number}			& \code {1 byte}		\\
					\hline
				\end {tabular}
			\end {center}
		\item Then, for example the record \code {\{"initials": "TB", "age": 30\}}, would
			be stored in memory as (spaces added for clarity)
			\begin {center}
				\code {01010100 01000010 00011110}
			\end {center}
		\item Where, \code {01010100} $ \mapsto$ \code {'T'}, \code {01000010}
			$ \mapsto$ \code {'B'}, and \code {00011110} $ \mapsto$ 
			\code {30}
		\item Suppose we tried to enter a middle initial, then we would expect the
			following record:
			\begin {center}
				\code {\{"initials": "TVB", "age": 30\}}
			\end {center}
		\item However, in memory this record would be stored as:
			\begin {center}
				{\color {green}\code {01010100 01010110 01000010}}
				{\color {red} \code {00011110}}
			\end {center}
		\item There were only supposed to be 3 bytes written into memory, but 4 bytes
			were received
		\item The first two bytes will be read in as the initals \code {"TV"}, and
			the byte that was meant to represent the initial \code {'B'} will be
			read in as the integer \code {66}
		\item The resulting record will be \code {\{"initials": "TV", "age": 66\}},
			and the byte {\color {red} \code {00011110}} will be written into some 
			memory address that the program wasn't supposed to access
		\item This could cause an error, if that memory address was being used by some
			other program
		\item However, if the memory address was not being used, then it is possible that
			no error would be thrown
		\item This could result in the error going unnoticed, and a program which
			inputs the wrong age whenever a person enters in a middle initial
		\item This type of error can cause issues later, and would be very difficult to
			debug
		\item These types of errors are exploited to create 
			\textbf {Buffer Overflow Exploits}, where a nefarious actor uses this
			undefined behaviour to insert nefarious code in place of some system code
			that runs with elevated privelages
		\item In this example, one could enter in an arbitary length string for the
			initials field followed by some nefarious code
		\item If the initials string is exactly the right number of bytes, then this
			could put the byte containing the first instruction of the nefarious code
			in the memory address where some procedure for a system call is stored
		\item Then when the system attempts to make the function call, the nefarious code
			is run instead with root privelages
		\item Going back to the previous customer record example, if we want to allow
			for variable length records, then instead of storing the records as a 
			continuous block in memory, we should store pointers to the memory address
			of each record
		\item Then we could even specify the length of the field as the first byte 
			(or bytes) of the field, which would allow us to decide the field length
			at field creation
		\item If we assume that the memory address of the record is \code {0x6dfed4},
			then the record could be described as follows (recall that a pointer is
			4 bytes long)
			\begin {center}
				\label {ref:pointer_data}
				\begin {tabular}{ll}
					\hline
					\textbf {Field Name} 			&	\textbf {Record Start Pointer}	\\
					\hline
					\code {customerNamePointer}		&	\code {0x6dfed4}					\\
					\code {customerAddressPointer}	&	\code {0x6dfed4 + 4 bytes}		\\
					\code {customerAgePointer}		&	\code {0x6dfed4 + 8 bytes}		\\
					\code {customerIdPointer}		&	\code {0x6dfed4 + 12 bytes}		\\
					\hline	
				\end {tabular}
			\end {center}
		\item We could also have some variable length data, and some fixed width data, in
			that case, we would only store the variable length data as a pointer
	\end {itemize}
	
	
	
	
	
	
	
	
	
	
\end {document}